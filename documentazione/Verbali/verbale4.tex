\documentclass[12pt,a4paper,titlepage]{article}
%va sempre messo article per "program documentation"
\usepackage[italian]{babel}
\usepackage[T1]{fontenc}
\usepackage[utf8]{inputenc}
\usepackage{titlesec}
\usepackage{hyperref}
\usepackage[a4paper,top=2cm,bottom=2cm,left=3cm,right=3cm]{geometry}
\usepackage{soulutf8,color}
\usepackage{graphicx}
\usepackage{emptypage}                     
% pagine vuote senza testatina e piede di pagina
%\usepackage{hyperref}                     
% collegamenti ipertestuali
\usepackage{fancyhdr}
% pacchetto per intestazione e pie pagina
\pagestyle{fancy}
\lhead{\includegraphics[width=0.2\linewidth]{../../LogoSWEgGroup}}
\chead{}
\lfoot{Verbale}
\rhead{27/02/2017}
\cfoot{}
\rfoot{Piergiorgio Danieli}
\renewcommand{\headrulewidth}{0.2pt}
\renewcommand{\footrulewidth}{0.2pt}
\begin{document}
\section*{Verbale n4}
\subsection*{Data e Ora}
\begin{center}
\textbf{27/02/2017 11:00}
\end{center}
\subsection*{Partecipanti Interni}
\begin{itemize}
	\item Sebastiano Marchesini
	\item Alberto Gelmi
	\item Piergiorgio Danieli
\end{itemize}

\subsection*{Argomenti Trattati}
Nella mattina di Lunedi 27/02/2017 c'è stato un breve incontro tra alcuni componenti del team SWEg group alla fine della lezione di ingegneria del software tenuta dal professor Tullio Vardanega. Sono stati discussi i linguaggi da utilizzare per lo sviluppo del progetto. Abbiamo deciso, tornando all'idea iniziale, di utilzzare Java per il back-end e invece Javascript per il front-end, non utilizzando quindi Jolie per tutto ma soltanto per l'interfaccia. Sono stati decisi alcuni dei design pattern che verranno utilizzati e anche una architettura di base che verrà approfondita nei prossimi incontri già fissati anche con gli altri membri del gruppo. Sono rimasti alcuni dubbi sull'utilizzo del framework Angularjs ed il relativo pattern; anche questi verrano sciolti nei prossimi incontri. Non sono state poste domande rilevanti.
\end{document}
\documentclass[12pt,a4paper,titlepage]{article}
%va sempre messo article per "program documentation"
\usepackage[italian]{babel}
\usepackage[T1]{fontenc}
\usepackage[latin1]{inputenc}
\usepackage{titlesec}
\usepackage{hyperref}
\usepackage[a4paper,top=2cm,bottom=2cm,left=3cm,right=3cm]{geometry}
\usepackage{soulutf8,color}
\usepackage{graphicx}

\usepackage{emptypage}                     
% pagine vuote senza testatina e piede di pagina

%\usepackage{hyperref}                     
% collegamenti ipertestuali

\usepackage{fancyhdr}
% pacchetto per intestazione e pie pagina

\pagestyle{fancy}



\lhead{\includegraphics[width=0.2\linewidth]{../../LogoSWEgGroup}}
\chead{}
\lfoot{Verbale}
\rhead{22/12/2016}
\cfoot{}
\rfoot{Piergiorgio Danieli}
\renewcommand{\headrulewidth}{0.2pt}
\renewcommand{\footrulewidth}{0.2pt}


\begin{document}

\section*{Verbale n�1}

\subsection*{Data e Ora:}
\begin{center}
\textbf{22/12/2016 16:00}
\subsection*{Luogo:}
\textbf{Torre Archimede 1bc45}
\end{center}

\subsection*{Partecipanti Interni:}
\begin{itemize}
	\item Tutto il team SWEg Group
\end{itemize}

\subsection*{Partecipanti Esterni:}
\begin{itemize}
	\item NetBreak (Team Studenti)
	\item Professor Tullio Vardanega
	\item Dott. Claudio Guidi 
\end{itemize}

\subsection*{Argomenti Trattati}
Nel giorno 22/12/2016 si � svolto un incontro nell'aula della Torre Archimede 1bc45
tra i gruppi che hanno deciso di svolgere il capitolato
C1 ed il (proponente/committente).\\
Si � svolto tramite skype ed aveva lo scopo di un primo incontro tra le parti e per una breve introduzione al linguaggio Jolie, che verr� usato per svolgere almeno una parte di progetto.\\
Il dott. Guidi ha introdotto per circa un'ora il linguaggio agli studenti mostrando qualche esempio.
\\
In seguito i due gruppi hanno avuto la possibilit� di fare qualche domanda al dott. Guidi per 
cercare di capire meglio le richieste e quindi svolgere al meglio il progetto.

\subsection*{Domande e Risposte}
\begin{enumerate}
	\item \textit{Bisogna definire un'architettuta comune ai due gruppi?}\\
	\hspace{35pt}La risposta � stata affermativa. I due gruppi (o alcuni membri di essi) dovranno quindi lavorare insieme per definire un'architettura comune.
	\item \textit{Chi sono gli utenti di questo API Market?}\\
	\hspace{35pt}Gli utenti sono sia i developer che caricano le API, sia gli user, cio� chi compra ed utilizza le API.
	\item \textit{Come gestire il gateway?}\\
	\hspace{35pt}In questo caso il dott. Guidi ha proposto di poter snellire il traffico
del gateway utilizzando dei microservizi messi a disposizione dell'utente.
	\item \textit{Cosa sono esattamente gli SLA?}\\
	\hspace{35pt}Sono i Service Level Agreement, cio� la modalit�
di erogazione della API (durata, numero di chiamate, traffico dati) e vengono gestite dal Market;
\end{enumerate}

\end{document}
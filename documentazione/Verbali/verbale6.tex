\documentclass[12pt,a4paper,titlepage]{article}
%va sempre messo article per "program documentation"
\usepackage[italian]{babel}
\usepackage[T1]{fontenc}
\usepackage[latin1]{inputenc}
\usepackage{titlesec}
\usepackage{hyperref}
\usepackage[a4paper,top=2cm,bottom=2cm,left=3cm,right=3cm]{geometry}
\usepackage{soulutf8,color}
\usepackage{graphicx}
\usepackage{emptypage}                     
% pagine vuote senza testatina e piede di pagina
%\usepackage{hyperref}                     
% collegamenti ipertestuali
\usepackage{fancyhdr}
% pacchetto per intestazione e pie pagina
\pagestyle{fancy}
\lhead{\includegraphics[width=0.2\linewidth]{../../LogoSWEgGroup}}
\chead{}
\lfoot{Verbale n6}
\rhead{01/03/2017}
\cfoot{}
\rfoot{Piergiorgio Danieli}
\renewcommand{\headrulewidth}{0.2pt}
\renewcommand{\footrulewidth}{0.2pt}
\begin{document}
\section*{Verbale n6}
\subsection*{Data e Ora}
\begin{center}
\textbf{01/03/2017 10:30}
\end{center}
\subsection*{Partecipanti Interni}
\begin{itemize}
	\item Team SWEg group
\end{itemize}
\subsection*{Partecipanti Esterni}
\begin{itemize}
	\item Team NetBreak
\end{itemize}
\subsection*{Argomenti Trattati}
In data 01/03/2017 si e' svolto un incotro tra i due gruppi che hanno scelto di svolgere il capitolato C1. Su suggerimento del proponente i due gruppi devono collaborare per avere una architettura di base simile. Questo infatti e' l'argomento principale trattato in questo incontro. C'e' stata una lunga discussione riguardo il back-end del nostro sistema. Dopo circa un'ora e mezza siamo arrivati ad una conclusione che entrambi i team approvano. Un altro argomento che e' stato trattato e' il database. Anche in questo caso il commitente ha suggerito di collaborare per avere una base di dati comune. L'incontro si e' concluso senza che ci siano altri quesiti da risolvere.
\end{document}
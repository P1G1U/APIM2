\documentclass[12pt,a4paper,titlepage]{article}
%va sempre messo article per "program documentation"
\usepackage[italian]{babel}
\usepackage[T1]{fontenc}
\usepackage[latin1]{inputenc}
\usepackage{titlesec}
\usepackage[hidelinks]{hyperref}
\usepackage[a4paper,top=2cm,bottom=2cm,left=1cm,right=1cm]{geometry}
\usepackage{soulutf8,color}
\usepackage{multirow}
\usepackage{lscape}
\usepackage{graphicx}
\usepackage{eurosym} %per il simbolo dell' \euro
\usepackage{caption}
\usepackage{subcaption}
\usepackage{url}

\usepackage{emptypage} 
% pagine vuote senza testatina e piede di pagina

%\usepackage{hyperref} 
% collegamenti ipertestuali

\usepackage{fancyhdr}
% pacchetto per intestazione e pie pagina

\pagestyle{fancy}
\usepackage{lastpage}


% PARTE DESTINATA ALLE NESTED BOXES:
\usepackage[usestackEOL]{stackengine}[2013-11-09]
\def\stackalignment{l}
\setstackgap{L}{1.4\baselineskip}
\fboxsep=4pt\relax


%rinomino table of contents
\renewcommand{\contentsname}{Contenuti:}
\renewcommand{\listfigurename}{Lista delle figure}
\renewcommand{\listtablename}{Lista delle tabelle}

\newcommand{\docs}{./sezioni/}
%posizione immagini per il documento corrente 
\newcommand{\docsImg}{./sezioni/img/}

\newcommand{\pedice}[1]{\raisebox{-0.6ex}{\scriptsize #1}}
%comando usato per il glossario


% \\ indica interruzione di riga

% compilate 2 volte per documenti con indice

% {\em qui testo in corsivo}
% {\bfseries qui testo in grossetto}

%LISTE NUMERATE
%\begin{enumerate}
%\item primo
%\item secondo
%\item terzo
%\end{enumerate}

%LISTE PUNTATE
%\begin{itemize}
%\item primo
%\item secondo
%\item \dots
%\end{itemize}

%TABELLA
%\begin{tabular}{|c|c|c|}
%indica una tabella con 3 colonne e pos. testo centrale. La barra verticale ( | ) indica che vi è una linea divisoria verticale tra le celle.
%\hline	linea separatrice orizzontale
%testo1& testo2& testo3\\
% & segna la fine del testo nella cella, \\ indica il fine della riga 

\newcommand{\minitab}[2][1]{\begin{tabular}#1 #2\end{tabular}}

%GRAFICI
%\begin{figure}
%\includegraphics{filegrafico}
%comando per includere le immagini (controllare i formati)
%\caption{didascalia}
%\label{nome}
%\end{figure}

%----------------------------------------------------------- INIZIO TEMPLATE TITOLO

\usepackage{xcolor} % Importa i colori per la prima pagina
\usepackage{fix-cm} % Permette l'incremento del font oltre misura

\newcommand{\HRule}[1]{\hfill \rule{0.2\linewidth}{#1}} % Horizontal rule at the bottom of the page, adjust width here

\definecolor{grey}{rgb}{0.9,0.9,0.9} % Colore del box del titolo

\begin{document}
	
	\thispagestyle{empty} % Toglie il numero della pagina nella prima pagina
	
	%----------------------------------------------------------------------------------------
	%	TITLE SECTION
	%----------------------------------------------------------------------------------------
	
	\colorbox{grey}{
		\parbox[t]{0.91\linewidth}{
			\centering \fontsize{50pt}{80pt}\selectfont % Il primo è la grandezza del font, il secondo lo spazio lasciato
			\vspace*{0.7cm} % Spazio dall'inizio del box al titolo
			
			\raggedleft
			\includegraphics[width=\linewidth]{../../LogoSWEgGroupSFONDOVUOTO}
			
			\hfill Specifica Tecnica \\
			
			\vspace*{0.7cm} % Spazio dalla fine del testo alla fine del box
		}
	}
	
	%----------------------------------------------------------------------------------------
	
	\vfill % Spazio dalla fine del box alle altre informazioni
	
	%----------------------------------------------------------------------------------------
	%	Informazioni sul documento
	%----------------------------------------------------------------------------------------
	
	{\centering \large 
		\hfill \textbf{Versione} 			1.0.0 \\		
		\hfill \textbf{Data di Rilascio}	XX/XX/2017 \\ 
		\hfill \textbf{Redazione} 			Gianluca Crivellaro \\
		\hfill								Piergiorgio Danieli \\
		\hfill								Sebastiano Marchesini \\
		\hfill \textbf{Validazione} 		Pietro Lonardi \\
		\hfill \textbf{Responsabile}		Alberto Gelmi \\
		\hfill \textbf{Uso} 				Esterno \\
		\hfill \textbf{Destinato} 			ItalianaSoftware S.r.l \\
		\hfill								Prof. Vardanega Tullio \\ 
		\hfill								Prof. Cardin Riccardo \\
		
		\HRule{1pt}
		
		\textbf{Sommario} \\
		Questo documento descrive la specifica tecnica e l'architettura del prodotto sviluppato.
		
	} % Linea orizzontale di estetica
	
	
	%----------------------------------------------------------------------------------------
	
	\clearpage % Parta bianca finale della pagina
	
	%----------------------------------------------------------- FINE TEMPLATE TITOLO
	
	\lhead{\includegraphics[width=0.2\linewidth]{../../LogoSWEgGroup}}
	\chead{}
	\lfoot{Specifica Tecnica}
	\cfoot{}
	\rfoot{\thepage\ di \pageref{LastPage}} %compilare due volte per avere il numero di pagine totali
	\renewcommand{\headrulewidth}{0.2pt}
	\renewcommand{\footrulewidth}{0.2pt}
	
	\rhead{Registro Modifiche}
	\section{Registro Modifiche}
	\small %rippicciolisce il testo
	
	{\renewcommand\arraystretch{1.2} %aumenta l'altezza di ogni riga
		\begin{tabular}{|l|c|c|c|}
			\hline
			{\textbf{Modifica}}&{\textbf{Nome}}&{\textbf{Data}}&{\textbf{Ver.}}\\
			\hline
			0.0.1 & Impostazione documento & Sebastiano Marchesini & 14/02/2017 \\
			\hline
		\end{tabular}
	}
	\normalsize
	
	\newpage
	
	\thispagestyle{empty}
	\tableofcontents
	%crea indice automaticamente
	\listoftables
	\listoffigures

	\newpage

	\rhead{Introduzione}
	\input{\docs Introduzione.tex}
	\newpage
	\rhead{Tecnologie Utilizzate}
	\input{\docs TecnologieUtilizzate.tex}
	\newpage
	\rhead{Descrizione Architettura}
	\input{\docs DescrizioneArchitettura.tex}
	\newpage
	\rhead{Architettura Front End}
	\input{\docs ArchitetturaFrontEnd.tex}
	\newpage  
	\rhead{Architettura Back End} 
	\input{\docs ArchitetturaBackEnd.tex}
	\newpage
	\rhead{Architettura Data Base}
	\input{\docs ArchitetturaDataBase.tex}
	\newpage
	\rhead{Design Pattern}
	\input{\docs DesignPattern.tex}
	\newpage
	\rhead{Diagramma Attività}
	\input{\docs DiagrammaAttivita.tex}
	\newpage
	\rhead{Tracciamento}
	\input{\docs Tracciamento.tex}
	\newpage
	%Appendice
	\rhead{APPENDICE - Descrizione Design Pattern}
	\input{\docs DescrizioneDesignPattern.tex}
	\newpage
	\rhead{APPENDICE - MockUp}
	\input{\docs MockUp.tex}
	%\newpage
	%\appendix
	
	%\bibliography{../bibliografia}

\end{document}
